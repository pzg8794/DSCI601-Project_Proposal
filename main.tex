\documentclass[12pt]{article}

\input{proposal-defs-alt}
\usepackage[margin=0.6in]{geometry}
\usepackage{url}
\usepackage{latexsym}
\usepackage{eepic,color,bm,array,amsmath}
\usepackage{enumitem}
\usepackage[hidelinks]{hyperref}
\usepackage{microtype}

\setlist[itemize]{leftmargin=1.25em, itemsep=0.05em, topsep=0.1em}
\setlength{\parskip}{0.15em}
\setlength{\parindent}{0pt}

\newcommand{\eg}{{\it e.g.}}
\newcommand{\ie}{{\it i.e.}}

\begin{document}

\pagestyle{empty}

\noindent
\begin{center}
{\large\bf Fair Contextual Bandits for Equitable Diagnostic Decision-Making Under Missing Context}
\end{center}

\begin{center}
\begin{minipage}{0.48\linewidth}
\begin{center}
{\bf Piter Z. Garcia Bautista}\\
\begin{small}
MS Data Science / Bioinformatics\\
Rochester Institute of Technology\\
{\it pizg8794@g.rit.edu}
\end{small}
\end{center}
\end{minipage}
\begin{minipage}{0.48\linewidth}
\begin{center}
{\bf Dr. Daniel Krutz, Travis Desell}\\
\begin{small}
Department of Software Engineering, Data Science\\
DSCI 601 Project Advisors\\
{\it dxkvse@g.rit.edu, tjdvse@g.rit.edu}
\end{small}
\end{center}
\end{minipage}
\end{center}

\noindent
This project will develop a practical, reproducible framework for {\bf contextual multi-armed bandits} (iC-/C-MABs)
to make sequential decisions under uncertainty and limited resources while treating {\bf algorithmic fairness} as a first-class objective.
The work on this generic framework focuses on measuring and improving fairness across the clinical (diagnostic-like) and quantum domains, through the following deliverables: i) a simulation-first diagnostic-like sequential decision environment with controllable context missingness/uneven measurement and distribution shift,
ii) a fairness aware and contextual MABs evaluation stack (baselines + contextual policies) with time-evolving group disparity reporting and at least one mitigation mechanism,
and iii) a transfer study demonstrating the same policy stack in a quantum-network routing + qubit allocation simulator.


\paragraph{Background:} \hspace{-5mm}
Many real-world workflows in both clinical diagnostics and quantum networking require sequential choices under uncertainty, resource constraints, and distribution shift. These settings can amplify inequities when some groups systematically have lower-quality context or different error profiles. This proposal frames this as a contextual bandit problem: at each step, choose an action (an ``arm'') given observed context to maximize utility while controlling fairness gaps. In both domains, multi-armed bandits formalize online decision-making with exploration--exploitation tradeoffs, and contextual bandits condition decisions on side information that can improve sample-efficiency and stability. In the quantum-network domain, context may include link-quality estimates, network load/queue signals, and qubit availability/quality, but these signals can be delayed, noisy, or partially observed. However, in clinical diagnostics, context may include patient features, test and sample-quality indicators, and operational constraints, but access to context can be incomplete or systematically noisier for some populations. These limitations create performance and fairness risks, especially when optimizing aggregate performance, which can hide subgroup error spikes (e.g., false-negative gaps) unless the system is explicitly monitored and constrained.

This proposal draws on contextual bandit methods for sequential decision-making with side information \cite{li2010contextual,abbasi2011linear} and on algorithmic fairness work that defines and measures group-based error disparities \cite{hardt2016equality}. Practically, it builds on an existing quantum-network routing + qubit allocation simulator and on prior work, {\it Equitable Bioinformatics} \cite{iste780_equitas}, operationalizing fairness audits and mitigation for diagnostic-relevant pipelines. The novelty is the {\bf integration}: evaluating fairness aware and contextual MABs policy choices in both a diagnostic-like sequential environment and a quantum-network routing/qubit-allocation environment, while reporting and mitigating fairness disparities {\bf over time} rather than only post-hoc.
Unlike prior work that often reports utility-only bandit performance or post-hoc fairness for static predictors, this project makes the time-evolving utility--fairness tradeoff explicit and tests transfer across both testbeds.


\paragraph{Scientific Merit:} \hspace{-5mm}
The core scientific question is: {\it when does informative context (and how it is modeled) materially reduce disparity and error in sequential decision-making under shift?} This is challenging in modern contextual-bandit settings because (1) feedback is partial and delayed (bandit feedback), (2) distributions shift, and (3) fairness constraints can conflict with pure utility optimization. The key innovative component is that a {\bf quantum-network routing and qubit-allocation environment} is treated as a first-class testbed: the same fairness-aware contextual bandit stack is evaluated in (i) a diagnostic-like simulation and (ii) a quantum routing + qubit allocation simulator to test robustness and transfer across radically different domains.


\paragraph{Broader Impacts:} \hspace{-5mm}
If successful, this work provides a concrete, reproducible framework for fairness-aware sequential decision systems in diagnostics and other public-health settings (including COVID-style test allocation) where both resources and context quality are limited. It also advances trustworthy learning-based control in quantum networking by adding explicit disparity monitoring and constraints to online routing/allocation policies, producing reportable utility--fairness tradeoffs under shift. For this project, the result is a reusable evaluation harness (two testbeds + shared policy API) that supports advisor-driven experimentation and future publication-quality benchmarking of fairness-aware bandit policies.


\paragraph{Approach:} \hspace{-5mm}
The goal is to implement and evaluate fairness aware and contextual MABs policies across two testbeds (diagnostic-like sequential decision simulation + quantum-network routing/qubit allocation simulator), and to report utility--fairness tradeoffs under shift. The work is structured to be feasible without sensitive clinical data access (simulation-first), while remaining extensible to approved open datasets. The tasks are:
\begin{itemize}
\item {\bf Build diagnostic testbed:} implement a sequential decision simulation (test/model choice, retesting, pipeline choice) with distribution shift and controllable context missingness/uneven measurement.
\item {\bf Build quantum testbed:} integrate policies into the quantum routing + qubit allocation simulator; define flow/node groups and track service equity across groups.
\item {\bf Implement bandit policies:} non-contextual baselines (epsilon-greedy, UCB, Thompson), a contextual baseline (LinUCB-style), and one informed contextual (iCMAB-style) policy under a shared interface.
\item {\bf Define reward + metrics:} utility with explicit cost/latency and safety weighting; report regret/utility plus time-evolving fairness metrics (group-wise FNR/FPR gaps and at least one additional criterion).
\item {\bf Add fairness mitigation:} implement one fairness-aware mechanism (constraint/penalty or calibration) and quantify utility--fairness tradeoffs across both testbeds.
\item {\bf Package for reproducibility:} Python 3.11+ modular scripts (not one notebook), fixed seeds + configs, and a minimal test suite (sanity checks for simulations and metric computation).
\end{itemize}
Phase 1 includes environment + baselines + initial fairness audit/mitigation; Phase 2 includes robustness under shift + ablations + packaging for code review and final report/demo.


\paragraph{References:} \hspace{-5mm}
\begin{thebibliography}{10}
\bibitem{ga_work_bandits}
P. Z. Garcia Bautista. {\it Qubit Allocation in a Quantum Network using Stochastic Bandits}. Internal manuscript (course/research draft), 2026.
\bibitem{iste780_equitas}
Garcia Bautista, P. Z. (2025). {\it Equitable Bioinformatics: Enhancing Diagnostic Decision-Making through RNA and Biomarker Data}. Unpublished paper, Rochester Institute of Technology.
\bibitem{li2010contextual}
L. Li, W. Chu, J. Langford, and R. Schapire. A contextual-bandit approach to personalized news article recommendation. In {\it WWW}, 2010.
\bibitem{abbasi2011linear}
Y. Abbasi-Yadkori, D. P{\'a}l, and C. Szepesv{\'a}ri. Improved algorithms for linear stochastic bandits. In {\it NeurIPS}, 2011.
\bibitem{hardt2016equality}
M. Hardt, E. Price, and N. Srebro. Equality of opportunity in supervised learning. In {\it NeurIPS}, 2016.
\end{thebibliography}

\end{document}
