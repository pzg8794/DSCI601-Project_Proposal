\documentclass[12pt]{article}

\input{proposal-defs-alt}
\usepackage[margin=0.6in]{geometry}
\usepackage{url}
\usepackage{latexsym}
\usepackage{eepic,color,bm,array,amsmath}
\usepackage{enumitem}
\usepackage[hidelinks]{hyperref}
\usepackage{microtype}

\setlist[itemize]{leftmargin=1.25em, itemsep=0.05em, topsep=0.1em}
\setlength{\parskip}{0.15em}
\setlength{\parindent}{0pt}

\newcommand{\eg}{{\it e.g.}}
\newcommand{\ie}{{\it i.e.}}

\begin{document}

\pagestyle{empty}

\noindent
\begin{center}
{\large\bf Fair Contextual Bandits for Equitable Diagnostic Decision-Making Under Missing Context}
\end{center}

\begin{center}
\begin{minipage}{0.48\linewidth}
\begin{center}
{\bf Piter Z. Garcia}\\
\begin{small}
MS Data Science / Decision-Making \& Algorithmic Fairness\\
Rochester Institute of Technology\\
{\it pizg8794@g.rit.edu}
\end{small}
\end{center}
\end{minipage}
\begin{minipage}{0.48\linewidth}
\begin{center}
{\bf Dr. Daniel Krutz, Travis Desell}\\
\begin{small}
Department of Software Engineering, Data Science\\
DSCI 601 Project Advisors\\
{\it dxkvse@g.rit.edu, tjdvse@g.rit.edu}
\end{small}
\end{center}
\end{minipage}
\end{center}

\noindent
This project will develop a practical, reproducible framework for {\bf contextual multi-armed bandits} (iCMABs/CMABs)
to make sequential decisions under uncertainty and limited resources while treating {\bf algorithmic fairness} as a first-class objective.
The work on this generic framework focuses on measuring and improving fairness across the clinical (diagnostic-like) and quantum-network routing domains through the following deliverables:
i) a simulation-first diagnostic-like sequential decision environment with controllable context missingness/uneven measurement and distribution shift,
ii) a fairness-aware contextual MAB evaluation stack (baselines + contextual policies) with time-evolving group disparity reporting and at least one mitigation mechanism,
and iii) a transfer study demonstrating the same policy stack in a quantum-network routing simulator.

Here, ``diagnostic-like'' refers to clinical diagnostic sequential decision workflows such as diagnostic test selection, triage or follow-up escalation, and model/pipeline selection under partial feedback and non-stationarity. It also includes cases where key diagnostic components (tests, instruments, or patient information) are unavailable or delayed due to resource constraints or allocation policies, creating group-dependent missing context.

Mitigation mechanisms considered include fairness-regularized policy updates (constraints/penalties), group-aware calibration or thresholding, and missingness-aware context augmentation (e.g., feature acquisition or imputation policies).


\paragraph{Background:} \hspace{-5mm}
Many real-world workflows in both clinical diagnostics and quantum-network routing require sequential choices under uncertainty, resource constraints, and distribution shift. These settings can amplify inequities when some groups systematically have lower-quality context or different error profiles. This proposal frames these workflows as a contextual bandit problem: at each step, choose an action (an ``arm'') given observed context to maximize utility while controlling fairness gaps. In both domains, multi-armed bandits formalize online decision-making with exploration--exploitation tradeoffs, and contextual bandits condition decisions on side information that can improve sample-efficiency and stability. In the quantum-network routing domain, routing decisions are constrained by probabilistic link success, limited entanglement/quantum-resource availability, and time-varying congestion; context may include link-quality estimates and network load/queue signals, but these signals can be delayed, noisy, or partially observed. In this project, service equity is defined as parity in routing outcomes, quantified by group disparities in success probability and latency across flow groups. In clinical diagnostics, context may include patient features, test and sample-quality indicators, and operational constraints, but access to context can be incomplete or systematically noisier for some populations. These limitations create performance and fairness risks, especially when optimizing aggregate performance, which can hide subgroup error spikes (e.g., false-negative gaps) unless the system is explicitly monitored and constrained. Unlike many evaluations that report utility-only bandit performance or post-hoc fairness for static predictors, this project makes the time-evolving utility--fairness tradeoff explicit and tests transfer across both testbeds.


\paragraph{Scientific Merit:} \hspace{-5mm}
The core scientific question across the clinical diagnostics and quantum-network routing domains is: {\it when does informative context (and how it is modeled) materially reduce disparity and error in sequential decision-making under shift?} This is challenging in both domains because (1) feedback is partial and delayed (only the chosen test/routing action reveals an outcome), (2) conditions shift over time (patient mix and operational processes; network load and link conditions), and (3) fairness constraints can conflict with pure utility optimization. This proposal draws on contextual bandit methods for sequential decision-making with side information \cite{li2010contextual,abbasi2011linear} and on the algorithmic fairness work that defines and measures group-based error disparities \cite{hardt2016equality}. Practically, it builds on an existing quantum-network routing simulator and previous work on {\it Equitable Bioinformatics} \cite{iste780_equitas}, operationalizing fairness audits and mitigation for diagnostic-relevant pipelines. The key innovative component is the two-domain evaluation: treating the {\bf quantum-network routing environment} as a first-class testbed alongside the diagnostic-like simulation, using the same fairness-aware contextual MAB stack in both to test robustness and transfer across radically different domains.


\paragraph{Broader Impacts:} \hspace{-5mm}
If successful, this work provides a concrete, reproducible framework for fairness-aware contextual sequential decision systems in both clinical diagnostics and quantum-network routing. In the clinical diagnostic domain (diagnostic-like test/model/retesting selection, including COVID-style test allocation), the framework supports measuring and mitigating disparities that can arise when resources and context quality are uneven across patient groups. In quantum-network routing, it advances trustworthy routing by adding explicit service-equity monitoring and disparity-aware constraints to online routing policies under probabilistic link success, time-varying congestion, and scarce quantum resources across user/flow groups. For this project, the result is a reusable evaluation harness (two testbeds + shared policy API) that supports advisor-driven experimentation and benchmarking of fairness-aware contextual MAB policies. This is a {\bf novel integration}: evaluating fairness-aware contextual MAB policy choices in both domains, while reporting and mitigating fairness disparities {\bf over time} rather than only post-hoc.


\paragraph{Approach:} \hspace{-5mm}
The goal is to implement and evaluate fairness-aware contextual MAB policies in two domains across two testbeds (diagnostic-like sequential decision simulation + quantum-network routing simulator), and to report utility--fairness tradeoffs under shift. The work is structured to be feasible without sensitive clinical data access (simulation-first), while remaining extensible to approved open datasets. The tasks are:
\begin{itemize}
\item {\bf Build diagnostic testbed:} implement a sequential decision simulation (test/model choice, retesting, pipeline choice) with distribution shift and controllable context missingness/uneven measurement.
\item {\bf Build quantum testbed:} integrate policies into the quantum-network routing simulator; define flow/node groups and track service equity across groups.
\item {\bf Implement bandit policies:} non-contextual baselines (epsilon-greedy, UCB, Thompson), a contextual baseline (LinUCB-style), and one informed contextual (iCMAB-style) policy under a shared interface.
\item {\bf Define reward + metrics:} utility with explicit cost/latency and safety weighting; report regret/utility plus time-evolving fairness metrics (group-wise FNR/FPR gaps and at least one additional criterion).
\item {\bf Add fairness mitigation:} implement one fairness-aware mechanism (constraint/penalty or calibration) and quantify utility--fairness tradeoffs across both testbeds.
\item {\bf Package for reproducibility:} Python 3.11+ modular scripts, fixed seeds + configs, and a minimal test suite (sanity checks for simulations and metric computation).
\end{itemize}
Phase 1 establishes baseline performance and group disparities under controlled missing/uneven context, evaluates an initial mitigation mechanism, and produces an initial demo; Phase 2 tests robustness under distribution shift, runs ablations, and packages the framework for code review and final report/demo.


\paragraph{References:} \hspace{-5mm}
\begin{thebibliography}{10}
\bibitem{ga_work_bandits}
P. Z. Garcia. {\it Quantum-Network Routing using Stochastic Bandits}. Internal manuscript (course/research draft), 2026.
\bibitem{iste780_equitas}
Garcia, P. Z. (2025). {\it Equitable Bioinformatics: Enhancing Diagnostic Decision-Making through RNA and Biomarker Data}. Unpublished paper, Rochester Institute of Technology.
\bibitem{li2010contextual}
L. Li, W. Chu, J. Langford, and R. Schapire. A contextual-bandit approach to personalized news article recommendation. In {\it WWW}, 2010.
\bibitem{abbasi2011linear}
Y. Abbasi-Yadkori, D. P{\'a}l, and C. Szepesv{\'a}ri. Improved algorithms for linear stochastic bandits. In {\it NeurIPS}, 2011.
\bibitem{hardt2016equality}
M. Hardt, E. Price, and N. Srebro. Equality of opportunity in supervised learning. In {\it NeurIPS}, 2016.
\end{thebibliography}

\end{document}
